\documentclass{article}
\usepackage[T1]{fontenc}
\usepackage{microtype}% Pour l'ajustement de la mise en page
\usepackage[pdfusetitle,hidelinks]{hyperref}
\usepackage[english]{french} % Pour les règles typographiques du français
\usepackage[series={},nocritical,noend,noeledsec,nofamiliar,noledgroup]{reledmac}
\usepackage{reledpar} % Package pour l'édition

\usepackage{fontspec} % Package pour mettre Junicode comme police (important pour MUFI)
\setmainfont{Junicode}[
Extension=.ttf,
BoldFont=*-Bold]


\begin{document}

\date{}
        \title{Edition numérique (partielle) du "Miroir des dames"}
\maketitle

\begin{pages} 
\beginnumbering
        S Ci cõmence le prologue seur le liure qui est apelez le miroir des dames. selonc ce que dit uns me- stres qui est nõmez uege- cius ou liure que il feit de ce qui apartient a cheuale- rie. Il fu acoustume enciẽ- nement bonne et seinne doctrine mettre en escript. pour offrir et presenter aus princes et aus granz seig- sieurs. quar nulle chose nest droitement encõmẽcee se eile nest premieremẽt a dieu plaisent et du prin-  confermee. Ne il nest nule \textit{per}sonne aqui il apar tiegne plus grant scien- ce et scipience. que au prin ce. de qui la doctrine doit a touz ses soubgiez profiter. la quel chose lempereur Octouien et les autres pͥn ces anciens garderent et pourchacierent. selonc ce que il est montre et declare es feis des empereeurs par plu- seurs exemples. Ce est la sentence du meistre dessus dit. ¶ Les paroles du quel qui bien entendroit et dili- gement peseroit il trouue- roit. que le temps encien fu de grant beneurte. au re gart du temps present. Quar adonc les princes estudioient par grant dili- \marginpar{.I. }gence es ars et es sciences. et auoient les bons clers en grant amour. et en re- uerence. Quar estude et science ne sont pas con- traires a cheualerie. ainz se sont touz iours entra compaignies. selonc les enciẽnes hystoires. Et ce nest mie merueille. quar cheualerie deffent clergie. Et clergie enseigne et a- drece cheualerie. ¶Et pour ce en toutes monarchies bien ordonnees estude et cheualerie ont touz iours este ensemble senz estre des- seurees. Dont tant cõme les caldes amerent estude iustice et clergie. tant il furent puissent et uertu- eus contre touz autres et orent \textit{per}fecte seignorie. ¶Ainssi lisons nous des Romains qui furent sei gneur de tout le monde nõpas seulement par for ce darmes. Mais par leur sens et par leur sauoir. ¶Derreĩnement par la pour- ueence et grace du Roy de̾ roys Ih̃ucrist ou Reaume de france ces .iii. choses des sus dictes ont regne lon guement. et seront Ius\textit{que}s a la fin du monde se il ni a empeschement par devers nous pour cause de nos pechiez. De quoy il est escripte Francia militibz gaudet Cest a dire que lonneur et la loenge de france. est en bons cheualiers. Ce est celle qui a acoustume peiz \textit{que}rir amer deffendre. norrir et soustenir. Et le sage du monde dit que la ou il a peiz et repos il y a sens et a⟦⟧st̾ prudence. ¶Ainsi lisons nous du Roy challemaig ne de qui la memoire ne doit point morir ne faillir. Que il fu mont feruent en lamour de sapience. Fondeeur de̾ estudes. Pere et promoteur des clers. et des estudiens. enseigniez et en- formez mont soufisẽment en leitres des latins et des grex. Ce fu cils qui les li- ures especiaument de la sainte escripture auoit gar- doit et souuent estudioit. Il nauoit pas mis en obli une parole de boece. qͥ bien feit a noter et a retenir. Et est la parole tele. que selonc la sentẽce de platon le bien cõmun et les Royaumes auroient grant prosperite et grant beneurte. Se les princes qui les gouu̾nent estoient apris et enlumine de la clarte de sapience. ¶Et pour ce que le sage Roy salomon dit. que la ou il na science qui apartient a lame. il ni a nul bien. pour tant tres noble et tres ex- cellent dame. Ma dame Iehanne Royne de france et de nauarre. considerãs. que tout ainsi que la pier- re precieuse assise en fin or est tres belle et tres replen- dissent tout aussi est il de uertu et de science assises en ame de noble et haute \textit{per}sonne. cõme sont Roys. Roynes. princes princesse̾. pour ce il li a pleu a moy petit et poure de lordre des freres meneurs cõmetre un petit liure moral et assez profitable. de latin translater en fransois et oece \marginpar{.ii. }metre. Le quel liuret puet estre apele le mireour des dames: a fin que elle sache uoaier et considerer cõment toute tache ostee de sa con- science. puisse estre bien or- donnee a dieu et a ce que a li apartient. Et cõment ou gouuernement de sa \textit{per}son ne. de son ostel et de ses soub giez elle se doit auoir. Et cõment auec touz senz nule reprehension doit honeste- ment conuerser. Et apres par quels merites puisse uenir a pardurable gloire. et sens fin auec le souue- rain roy regner. Ci est mis le fondement de leuure ensui- Salomons -ent qui fu de sa- pience par le don de dieu clerement enluminez en .i. sien luire qui est apelez pro- uerbes dit en tel maniere. Sapiens mulier edifi- cat domum suam. ¶Une chacune creature de sa con- dicion naturele. qui li est de dieu empreinte et dõnee. desierre la conseruacion de son estre. tent naturelmẽt au lieu ou elle est gardee et sauuee. \textit{et}\textit{que}nt elle uient au lieu de sa ꝯseruacion; illeuc prãt son ropos. Ce ueons \textit{nous} cleremẽt \textit{et} a\textit{per}temet en bestes mues qͥ quierent leur fosses et leur tanieres pour habi- ter et reposer. Et les oyse- aus du ciel ueons nous leur niz par grant et mer- ueilleus artifice feire et ede- fier. Ainsi le dit nostre sau- ueeur en leuuangile saint luc. Uulpes foueas habẽt et uolucres et .c̾. Et dauid S .c.xiiii. .c.ix. en son psautier dit. passer inuenit sibi domum et tur- tur nidum. Se il est donc ainsi que les creatures qͥ sont senz entendement et senz raison. quierent et edi- fient a si grant diligence le lieu de leur repos; par plus fort raison. hõme et fẽme creatures raisonna- bles dignement faites crees et formees a lymage et a la semblence de la be- noite trinite pour sa con- seruacion paiz et repos se doit porueoir. de maison bonne seure et couuenable pour y demorer et reposer. ¶Et pour ce le saint esperit a la loenge et cõmendacion de sage dame et bien pour- ueue dit la parole presente. Mulier sapien̾ edificat domum suam. Cest a dire la sage dame edifie sa mas- son pour sauuer lame. Et en ceste parole toutes fem mes generaument et en especial toutes granz da mes. Et singulierement celle qui est Royne doit considerer quele elle est de sa propre condicion. Quar elle est fẽme. et ce emporte ce mot. Mulier. Quele elle doit estre par acquisici- on. quar elle doit aquerir sens et sapience. Sapien̾. Cõment elle se doit occuper en bonne o\textit{per}acion quar elle doit edifier. sa meison. Edificat do.i. Et selonc ces .iii. choses. puet estre ce liure deuise en trois partie̾ principaus. La premiere partie contient .iii. conclu- sions ueritables en gen̾al \marginpar{.iii. }La premiere conclusion est cõment sage dame et royne doit diligẽment penser la condicion de sa nature. Secondemẽt la promocion de sa fortu- ne. Tiercement la \textit{per}fectiõ qui li est dehue cest grace diune. La condicion de nostre nature. En considerant la con- dicion de nostre nature. nous trouuerons matie- re et cause de nous hu- milier. et de nous po pri- sier. A quoy est necessaire deuant toutes choses: a- uoir cognoissence de soy meĩmes. Premier chapitre auoir cognoissence de soy meisme. Se il est donc ainsi que tu meites ton cuer et testude a auoir cog- noissence de toy; premiere- ment tu auras matiere nõpas de toy enorguillir. Mais de toy petit prisier. Et a ce nous amoneste Saint bernart. qui dit en tel maniere. Ie te conseille \textit{que} ta consideration cõmence a toy cognoistre a ceste fin que tu ne te occupes mie pour neãt toy laissie mis en obli. et a nonchaloir que te profiteroit il se tu gaai- gnoies tout le monde. et toy tout seul tu \textit{per}doies. Quar se il auenoit que tu cegneusses touz les mi- steres et secrez de dieu la hauteice du ciel. le le de la terre. le parfont de la mer. Se tu nas de ton estat co- gnoissence tu ressemble- ras celi qui edifie senz fon- dement de qui ledifice trait a ruine et ne se puet sou- b̾na stenir. Donc ce que tu edi- fieras senz auoir de toy cognoissence sera semblãt a la poudre qui est portee au uent. ¶Et doiz sauoir que \textit{per}sonne qui ne cog- noist soy et son estat res- semble les bestes mues qͥ nont sens. raison ne enten dement. De quoy parle le roy dauid en son psau- tier. qui dit en tel manie- re. Homo cum in honore esset non intellexit. \textit{et} c̾. Cest a dire que \textit{per}sonne cõbien quelle soit en grant estat et en grant dignete. se elle ne se cognoist et en tant. res- semble les bestes mues qui nont ne sens ne sapi- ence. Et ce li auient pour ce que il a. deffaut de cog- noissence si cõme dist est. ¶Et combien que sauoir soy cognoistre soit en tou- tes \textit{per}sonnes requis: toute uoie plus en princes et princesses. Quar le prince est chief de touz ses soubgi- ez Selonc ce que tesmoigne Hugues de Saint uictoir ou liure des sacremenz. ter- riẽne puissence dit il a le Roy pour chief. Et la puis- sence esperituele. a pour chief le pape. ¶Or est il ainsi \textit{que} les yex sont naturelment assis en la partie souuereĩ- ne qui est le chief. par les les quiex est entendue cog- noissence. pour quoy il sen- suit que en prince doit estre plus clere cognoissence \textit{que} es soubgiez. Et seroit cho- se monstrueuse \textit{et} ꝯtͤ nateͬ \textit{que} les yex ne feusent pas mis ne assis en leur chief. Aus- ci ce est chyse mont desaue- \marginpar{.iiii. }Hugues. nant et a reprouer. que le prince ait deffaut de cogno- issence. ¶Et pour ce lisons nous du roy nabugodo- nosor. que une des cause̾ principaus pour quoy il fu botez hors de son royau- me et mis entre les bestes mues. et bestes sauuages par lespace de .uii. ans. Fu pour ce que il ne se cognois- soit pas ne humilioit de- uant son createur. ¶Et a ce montrer a\textit{per}tement dit la sainte escripture. que quant il recoura son sens et sa cognoissence il fu resta- bliz et remis et son royau- me a grant honeur cõme deuant. Ceste hystoire est escripte plus pleĩnement ou liure de daniel le pro- phete. ou quart chapitre. ¶Et deuons entendre que que princes qui deffaut a de cognoistre soy et son estat nest pas seulement semblanz a bestes mues. Mes doit estre meins p̾siez. selonc ce que dit boeces en son secont liure de consola- cion ou il dit en tel manie- re. La condicion humaĩne creature est que tent seule- ment hõme est plus excel- lent de toutes autres chose̾ quant il se cognoist. Mais quant il se leisse a cognoi- stre. les autres creatures le seurmontent en dignete. Et rent la raison quar \textit{que} les bestes ont ignorance ce leur uient de leur nature. Et ignorance en hõme ui- ent de uice et de pechie. ¶Et pour ce nous enseig- ne profitablement. Saint bernart. qui dit ainsi. estu- boec b̾nart die soigneusement a auoir de toy cognoissence. quar tu seras meilleurs. et feras plus a loer se tu te cognois. que se toy leissie et en ne- gligence mis. tu cognois soies les cours des estoiles. les forces des herbes. les cõ plexions des hõmes. et a- uoies toute la science des choses du ciel et de la terre. ¶Et saint augustin en son liure de confessions deman- doit a nostre seigneur cog- noissence de soy et apres cognoissence de dieu. Do- mine inquit nosse me et nosse te . Secont chapitre. La seconde chose qui nous doit esmouoir a nous humilier et po pri- sier. est considerer lordure et la uilte de nostre poure na- ture. La quele est manife- stee et montree en .iii. choses par especial. quant ap̾sent. ¶Premierement en la ma- tiere de nostre premiere cre- acion et condicion de la \textit{que}le tesmoigne la sainte escrip- ture Genesi̾ .iiͦ. que le p̾mier hõme adam nostre p̾miers peres fu formez quant au corps nõpas seulement de terre. qui est le plus uil de touz les elemens. Meis fu formez du limon de la terre. qui nest autre chose que boe. et ordure. pour ce dit lescrip- ture ou secont chapitre de genesi. Aormauit deus hominem de limo terre. et c̾. Et Iob disoit a nostre seig- neur en soy humiliant. Memento queso quod sic̃ lutum feceris me. \textit{et} c̃. Cest a dire tres douz diex souieg- ne toy que tu mas feit et \marginpar{.u. }.c.x. forme dordure et de boe et a la fin tu me ramenras en poudre et en cendre. ¶Et ainsi apert euidẽmẽt la uil matiere de nostre premiere formacion. qui nous aprent a nous hu milier . Secondement. consi- derons la matiere de nostre concepcion et la matiere de la quele nous sõmes norri auant nostre natiuite. Et adonc troue- rons en nous plus grant uilte et ordure. Quar la matiere donc nous sõmes tuit conceu et norri sens exception. quar nuls non est franchiz ne exceptez. tant soit riches ne nobles. Celle matiere donc est tres puant horde et abhomina- ble. Selonc ce que disoit Iob ou .xiiͤi. chapitre. Qui̾ potest facere mundum de ĩmundo conceptũ semine. Cest a dire que nous qui sõmes hort de nostre con- ception. ne poons estre ne- stoie. se ce nest seulement par la grace de dieu. ¶En- ten diligẽment. et pense la tres grant uilte de la ma- tiere donc tu es conceuz. Quar lame de toy qui est pour cause de sa creacion tant digne tant noble tant bele et tant pure cõ- me celle qui est formee a lymage de dieu Si tost \textit{que} elle est iointe au corps for- me et conceu de ceste uil matiere. elle prant \textit{et} recept en soy la leideur lordure \textit{et} la pueur de pechie originel. par le quel elle est priuee de la uision de dieu \textit{per}dura- blement. Se elle nest auẽt lauee et nestoiee par la re- generacion du saint bap- tesme que nous receuons. La tierce chose qui uaut a nous hu- milier est consider les ordures. qui en nous sõt. et qui de nostre poure corp̾ issent. Quar se tu penses parfondement. lordure et la uilte qui ist de ta bou- che de ton nes. et des autre̾ parties de ton corps. tu ne ueis onques plus hort ne plus puant fumier. ¶Nous ueons bien que les herbes et arbres me- tent hors fleurs soef flai renz et odorans fueilles uerdoianz. et fruiz delitanz. Et noz fleurs sont puces lentes pooulz et autres meintes ordures et cheti- uetez. Si cõme il apiert par experience. Des arbres nous uiẽnent .uin. oile. baume et autres precieu- ses liqueurs. Et de nous issent liqueurs toutes pu- anz et pleĩnes de abhomi- nacion et corruption ¶Et ainsi apert la condicion de humeĩne nateͬ qͣnt a sa grant uilte. .iii. chapitre. Apres pour nous humilier et po pri- sier. deuons nous par gͣnt soing et auis penser a plu- seurs miseres pouretez et chetiuetez que nous auon̾. Et en y a quatre en genͬal. ¶La premiere est naturele. la quele nous auons et portons de nostre natiuite. et des ce que nous entron̾ en ce monde. ¶En signe de ce. si tost que nous sõ \marginpar{.ui. }mes ne. nous cõmenson̾ a gemir et a plorer. en mõ- trant le mechief et la mi- sere qui nous est a aue- nir. Et pour ce dit saint augustin \textit{que} le cõmencem̃t des enfanz. est plorer. ne ne puet estre que il rie ou fa- ce ioie pour le temps que il est nez. Mes plore et crie tout ainsi cõme il feust prophetes de la doleur que il doit apres endurer \textit{et} sou- frir. Et ce est une merueille. quant senz parler il prophe- cie ¶Et de ceste misere nũs tant soit riche tant soit gͣns tant soit noble. ne se puet franchir ne excepter. ¶Et pour tant le roy salomõ combien que il feust tres riche tres puissenz et no- bles. qui bien sauoit la uerte de nostre cõmune natiuite disoit de soy meis- mes ou liure de sapience. Iay dit il mis hors en plo- rant la uoiz semblant au̾ autres. qui est la uoiz de lermes. Et rent tantost la cause quar selonc ce \textit{que} il dit. Onques nũs Roy not autre cõmencement de natiuite. ¶Et certeĩne- ment cest tres grant mise- re. que de nostre natiuite quar nous naissons touz senz science senz parole et senz langage. Senz nule uertu. senz robes et sens couerture. .c.ix. ¶Or pense donc prince et princesse. Roy et royne la maniere et la conditiõ de ta natiuite. quar tu en- trez en ce monde dolenz et trites. poure̾. feibles et nõ sachens. et a mont petite difference entre toy et les bestes mues. Mes quant a aucunes nous auons meins que les bestes. Les bestes. si tost que nees sõt peuent corir et aler. hõme et fẽme mouoir ne se puet ne de ses piez aler nes a layde de ses mains ne se puet il mouuoir en soy traynant seur terre si cõ- me font aucunes bestes. ¶Et ueons aucunes foiz qui est greigneur mechief et misere aucuns naistre. en tres grant deformite et tres honteuse au quiex il eust miex este pͬueu. sil neussent onques este des hõmes ueu. quar aussi \textit{que} se il feussent montre a grãt deshoneur en les montre. ¶Pluseurs en y a. qͥ ont defaut de leur sens. et de leur membres et mont dau- tres deformitez en grant honte et reproche tristece \textit{et} doleur de leur amis et lig- nage. ¶Et pour ce que dit est. que nous naissons senz uesteure. pour ce dit Iob. Nudus egressus sum ex utero matris mee. \textit{et} c̃. Cest a dire. Ie suis issus touz nuz du uentre ma mere et touz nuz me couuient retorner. Et senz doubte nous naissons nuz cor- \marginpar{.uii. }porelment. Et nuz esperi- tuelment quar lame de nous est nue et uuide de uertu et de science. Et tout aussi que hõme et fẽme entre poure et nuz en ce monde. tout ainsi il cou- uient il departir. ¶Et plu- seurs en y a donc cest grãt doleur qui sen uont et de- partent plus nuz et plus poures que il ni entrerent. quar cum il soient en ta- chiez et enuelopez de plu- seurs pechiez et nuz de tout bien. rien nen portent de ce monde. fors \textit{que} leurs pechiez .iiii. Chapitre. La seconde misere que nous auons puet estre apelee contem- porele. Et proprement puet estre nõmee par tel langage. Quar elle \textit{nous} acompaigne par tout le temps de nostre uie ¶Et ceste misere aggrieuent .iiii. choses senz les quiex nous ne poons ceste uie p̾sente passer. ¶La premiere est nostre fragilite et febleice. la quele nous grieue for ment. De la quele parle. Saint gregoire en exposẽt cele parole de Iob. Homo natus de muliere breui. \textit{et} c̾. Et dit en tel maniere. Se nous considerons brief- ment tout ce qͥ est ci fait en \textit{nous} et de nous est peĩne et misere. Quar seruir a la corruption de nostre char. a ce qui li est de necessite. \textit{et} qui li est ostroie nest fors que misere. Auons nous froit il nous faut couurir et uestir. Auons nous faim. il nous faut viãde .c.xii. pour nous norrir et sou- stenir contre la chaleur \textit{nous} querons le froit. Et la san- te du corps. La quele gar der couuient par grant cautele. La quele ainsi gar- dee est ligierement perdue et dedanz brief temps. Et quant \textit{per}due est a grant peĩne et labeur est recouree. Et se il auient quelle le soit ne sõmes nous certein ne aseur de lauoir longuem̃t toutes ces choses ne montrẽt elle̾ pas euidẽment nostre misere et grant pourete. Aussi cõme se il uousit dire. certes oil. Et dit apres. nest ce dit il grant misere de hõ- me et de fẽme. qui \textit{per}du a son pays et en est botez hors. qui se delite et prant son solaz et son deduit en exil. De ce que il est formant greuez de diuerses cures \textit{et} labours engoisseus. Et nõ pourquant cõme negli- gens il dissimule a y pen- ser. Et enquor que il qui est priuez de la clarte par- durable. ne se donne gar de de locurte de son auugle- ment quele autre chose est dit il fors que misere qui est nee de nostre peĩne. ¶Et pour ce le roy dauid toutes ces choses conside- rans disoit en soy deuant dieu humiliant. Miser factus sum et curuatus usqz in finem contristatus ingrediebar. Cest a dire ie suis creature poure cheti- ue et miserable touz les iours. enclinez iusques a la terre. et ce est cause a moy de grant doleur et de grant tristeice. ¶De ceste \marginpar{.viii. }misere parle iob mont pro- prement qui dit ainsi. Homo natus de mulie- re breui uiuens tempore re pletur multis miseris. \textit{et} c̾. ¶En exposent ceste parole dit Saint bernait. hõme dit il nez de fẽme. de qui la uie est mont brief \textit{et} courte. est rempliz de mont de mi- seres. Et certeĩnement de meintes et multipliees. De misere de corps. de mi- sere de cuer. de misere en dor- ment en ueillant. et en tra- ueillant et de misere quel\textit{que} part il se uueut tourner. ¶Et ainsi apert clerement la misere de nr̃e fragilite. La seconde misere pour nous humi- lier. est temporele aduersite. De la quele nous sõmes continuelment afflit \textit{et} mo S. Bernart leste. et en y a pluseurs \textit{que} nous sentons en nous souuent par experience. De quiex dit Innocent pape. en .i. liure que il fait de la uilte de condicion humein- ne. en tel maniere. O dit il cõme hõme mortel est mis en grant angoisse. en gͣnt soing et en grant cure qui le moleste. en grant poor. donc il est espoentez. Doleur le tourmente. tristece le trouble. turbation le fait triste. Le poure et le riche. le sergent et le seigneur. Le marie et le continent. Le bon et le mauues. Et a bri- efment parler touz gene- raument des aduersitez mondaĩnes sont tormẽ- tez. Qui est il ou monde. qui onques ot un seul iour entier. en solaz et de- duit. en ioie et delit sens aucune aduersite. Qui est celi le quel aucun domage. offense et desplaisence. ou aucune passion nesmeuue en aucun temps. Qui est celi qui onques ne fu trou- blez de chose que il oist que il ueist. que en li deist. ou que en li feist. ¶Veritez est que touz iours apres hu- meĩne ioie et leeice. sensuit doleur aduersite et tristece. Bien le sauoit et esproue lauoit le sage roy Salo- mon. qui disoit en son li- ure qui est apelez prouer bes. Risus dolore misce- bitur. Cest a dire. que ris ioie et consolacion est mes- lee en doleur tristece et deso- lacion. et a la fin de ioie ist tristece et pleur. quar soub- deĩnement. quant pas ny pensons. ne garde ne nous donnons et de rien ne \textit{nous} doubtons uient un gͣnt mechief. une maladie et souuent la mort que nũs ne puet fuir ne eschaper. ¶Et pour tant disoit le sa- ge salomon en prouerbes. De glorieris in crastinũ ignorans quid su\textit{per}ueniẽs pariat dies. Cest a dire. tu ne te doiz mie glorifier ne toy trop atendre et fier ou temps a auenir quar tu ne puez sauoir quel chose ten auenra. ¶Apres les aduersitez qui nous uien- nent de nous ne sont pas seules. Mes en auons plu- seurs autres qui nous touchent mont souuent et tres grieues. de par \textit{nous} amis. Des quiex les mi- seres et aduersitez nous \marginpar{.ix. }.c.xiiii. ueons et esprouons souuẽt et mont nous tuichent au cuer et engendrent en nous grant tristece et do- leur. O dit un docteur cõ- ment sõmes nous angois- seusement trouble. dolereu- sement espoente quant nous sentons les mes- chiez et les domages de nos amis. qͣnt nons ueõs nos parenz et prochiens estre en peril. et nous sõ- mes en doubte de leur pe- ril. Qui est celi qui a le cuer et la uolente tant dure. cõme est pierre fer ou acier. qui nest esmeuz a plorer soupirer et lar- moyer. quant de son pro- chien parent ou ami il uoit la mort ou la grief maladie. ¶Nostre sau- veeur lh̾s selonc ce que il est escript en leuuangile saint Iehan quant il uit marie magdalene plorãt pour la mort de son ferere. ne se contint point que il ne plorast. Infremuit spiritu turbauit semetip̃m et lacrimatus est. Ne nest pas a entendre que il plo- rast seulement pour la mort du ladre. Mes prin- cipaument pour ce que celi qui mort estoit. aus miseres de ceste uie presen- te il rapeloit et reuenir fei- soir. ¶Et a declarer plus a\textit{per}tement lauersite de ce monde. esprouons nous aucune forz que le temps. qui pour reposer nous est donnez. cest a sauoir qͣnt nous dormons. Ne nous est pas de repos. Mes de travail et de labeur. quar table souuent en dormant et son gent nous sõmes espoentez. ¶Et combien que ce qui nous apert par songe en nostre dormant en uerite ne soit pas chose tristable ou espoẽ- toute uoie selonc la uerite les dormans se treuuent affliz lassez tristes et espo- entez. En signe de la quel chose aucune foiz il crient a haute uoyz en dorment \textit{et} se treu- uent plorãz \textit{et} touz effreez. La tierce misere et ad- uersite pour toy po prisier est enfermete corpo- rele. Et en y a pluseurs qͥ nous font granz angois- ses. ¶Qui est la \textit{per}sonne qui porroit penser trouuer ne soufisẽment dire ou ra- conter les manieres des maladies. les diuersitez des passions des dolours et des afflictions. \textit{que} nous sentons. ¶En uerite ie ne croy que il soit hõme ne clere uiuent tant sache de arismetique. qui est la sci- ence de nombrer qui le pu- isse sauoir ne trouuer for̾ celi qui scet le nombre et les nons des estoiles. qui est diex le tout puissent. ¶Quar senz doubte nous esprouons en nous plu- seurs differences denferme- tez et de maladies que il na en nostre corps de mẽ- bres. ne en nos membres de parties. tant soient peti- tes. Ne touz les phisiciens qui des le cõmencement du monde ont este. Ius\textit{que}s au iour dui. ne ceus qͥ serõt iusques a la fin du mon- de. onques ne porent en cerchier ne trouuer ne ia \marginpar{.x. }trouueront. ¶Or ueous chose mont merueilleuse. et nõpas seulement mer- ueilleuse mes miserable \textit{et} doloreuse que hõme pour qui fu cree et formee toute autre creature est soubmis et soubgiez a pluseurs peĩ- nes passions maladies \textit{et} enfermetez. touz seuls que ne sont toutes les autres creatures ensemble. ¶Et tout a ordonne le createur pour ce. que il se humilie deuant li et po se prise. La quarte misere que nous auons. est de noz enemis lassaut et impugnation des quiex nous sõmes continuelmẽt et \textit{per}illeusement assailliz qui pourchacent et quierẽt nostre mort et \textit{per}dicion. ¶Et sont quatre. lẽnemi. le mauues hõme. nostre propre corps et le monde des quiex aucune chose di- rons. ¶Lenemi nous assaut. auec les uices. selonc ce \textit{que} dit. Saint leon pape. en .i. sermon que il feit de la cir- concision. ¶Non desinit an- tiquꝰ hostis ubiqz deceptio num laqueos tendere. et c̾. ¶Il uueut ainsi dire en fransois lencien enemi sef- force touz iours tendre les laz pour nous deceuoir. et pourchace par grant instã- ce cõment la foy des cresti- eus il puisse corrumpre \textit{et} destruire. Il set mon bien. a quelx gens par sa mau- uaise suggestion il meite au deuant lardeur de cou- uoitise. Les quiex il deceiue par gloutõnie. Aus quiex il meite audeuant lambra sement du pechie de luxure. si cognoist en quiex cuers il puisse espendre le uenin den- uie. Il set mont bien les quiex il puet de peeur acra- uenter. Les quiex de doleur troubler les quiex de fausse ioie deceuoir. Les quiex \textit{per} merueilles et merueilleuses nouuelles assoter. de toutes \textit{per}sonnes il enquiert les cõ- dicions et les coustumes. Il pese les occupacions \textit{et} les cures. Il encerche les affec- cions et les uolentez. Et la ou il uoit chascun plus occupe il pourchace toutes les uoies et les achoisons de feire plus grant nuise- ment. et plus grant do- mage. ¶ Et pour ce nous deuons mont ueillier et soigneusement nous gar- der contre les frandes decep- cions et assaus de lenemi. De qui le barat nũs ne puet soufisẽment dire ne raconter. De qui la puissẽce en ne puet autre qui soit creee cree comparer. Et de qui la cruaute est tele que nũs ne la puet saouler. ¶Cest le dragon du quel il est escript en lapocalipse qui auoit sept chies et .x.cornes. Et proprement il a .uii. chies et .uii. testes quar il tempte de̾ .uii. pechiez mortelz. Et auec ce il a cornes en nom- bre de .x. quar il de tout son pooir sefforce de nous faire trespasser les .x. com̃de mens de dieu. ¶Apres \textit{nous} sõmes assailli et bien sou- uent et perilleusement des hõmes peruers et mauues. Les quiex aucunes foiz \textit{per} beles et douces paroles a \marginpar{.xi. }.c.xvi. pechie nous attraient. Au- cunes foiz par promesses. autre foiz par espoenter. Au- cunes foiz par fraudes en deceuant. Aucunes foiz \textit{per} fausses paroles et menson- ges pour nous peruertir. aucunes foiz par uitupe- res. uilenies et reproches. pour nous esmouoir. a corroz et impacience. Au- cune foz en nostre doma- ge pourchacent. Et aucu- nes foiz en nous de aucun bien retraient et esloignãt. ¶ Et toutes ces manieres de \textit{per}secucions. sentons \textit{nous} souuent auenir par ceus. qui de plus pres nous a- partiẽnent. ¶Et pour ce dit le prophete Ieremie. Un̾ quisqz a proxino suo se custodiat. \textit{et} c̃. Il uueut dire que qui bien se uueut garder de touz se doit garder de son uoisin. Et en son ꝓ- pre frere ne mete pas sa fiance. Et rent la cause. quar souuent auient que de noz propres freres \textit{nous} sõmes supplante et engi- gnie. Et ceus qui se mõtrẽt estre noz amis nous font fraude et barat. ¶ Et le pro- \textit{per}hete micheas dit ainsi. Nolite credere amico no- lite confidere induce. et c̾. Gardez dit il. que uous ne creez a ceus qui se dient estre uoz amis. Et ne me- tez pas nostre fiance ne uostre esperance en touz ceu̾ qui nous prometent con- duit. Et te garde bien de dire ton secret. a cele qui dort en ton sein en .i. meis- me lit. Cest a entendre qͣnt tu te doubtes raisõnablemẽt c.ix. et par grant presumption par les choses autre foiz es- prouees que ton secret po- int ne garderoit senz le re- ueler. Et rent le prophete et assigne la cause. quar souuent le fil fait uilenie a son pere. Et la fille seslie- ue par orgueil contre sa me- re. Et bien auient que ceux qui sont plus pres et plus priue. sont plus fort \textit{et} plus perilleus enemi. quar il sont tel qui sont enemi couuert et pour ce il fierent a descouuert. ¶Et tiex ma- nieres de gens aucunes foiz sont ueuz demorer et repeirier es courz meisons. et hostier Royaus. ¶ Et se tu demandes quiex gens ce sont. Ie te respon brief- ment que ce sont flateeur lobeeur manteeur. Des quier nũs prince ne se puet uenter que il nen ait en sa court pluseurs combien que il ne soient pas cogneu. ¶ De ceste ma- niere de gent autre foiz Le tiers enemi qui nostre propre char que nous assaut est nous portons. De la \textit{que}le nous sentons continuel- ment rebellion selons ce que dit. Saint pol en les- pitre que il fait a ceus de galathas Caro concu- piscit aduersus spiritum \textit{et} c̃ Il uueut dire que il a touz iours guerre entre le corpz et lame la char et lesperit. ¶Ceste guerre. de tant \textit{que}lle nous est plus prochiẽne. de tant est plus perilleuse quar ñs ne puet nuire dirons. \marginpar{.xii. }.c.v. ne greuer plus legieremẽt. que. feit lenemi priue et fa- milier. Cest un enemi que nous portons touz iours auec nous sens le quel nous ne poons estrie. ¶ Cest aduersaire nous aĩmons naturelment gardons et norrissons selonc ce que dit expressement. Saint pol en lespitre que il feit a ceus de ephese. Nemo carnem suam un\textit{que}m odio habuit sed nutrit et fouet ea \textit{et} c̃. ¶ Ih̾e tres douz diex cõme ci a grant et grief seruage et condition tres miserable. quar se nous nostre char norrissons. nous armonᷤ et fortifions nostre aduer- saire contre nous. Et qui ne le norrist il se mest a mort. occist et perist. ¶ Que poons nous et deuons fei- .c.v. re. Nous la deuons petite- ment soustenir. senz nul outrage. Durement et rigo- reusement chastier et disci- pliner. en la seruitute de les- perit ramener a ceste fin \textit{que} sa rebellion soit abaissee et quelle soit auec lesperit en bonnes euures exercitee et acoustumee. ¶Ainsi le fai- soit Saint pol en lespitire premiere que il feit a ceus de corinthe. ¶Castigo cor- pus meum et in seruitutẽ redigo. \textit{et} c̾. Ie chastie dit il mon corps et le meit a ser- uir lesperit. ¶De cest ene- mi parle. Saint bernart et dit. quel chose est nostre char. fors que. escume ue- stue de beaute fraaille et tost passant. Et sera asses tost une charoigne porrie et puant et uiande auers. Ceste poure ¶Ceste poure et chetiue cha- roigne. en tant est rebelle a lesperit et contraire que Saint pol. en lespitre aus romains se compleint aus- si cõme en plorant et dit Scio quia non habitat in me. hoc est in carne m̃a bonum. et c̾. Ie suis dit il certeins. que en ma char biens point ne habite. qͣr combien. que ie uueille au- cunes foiz bien faire Ie ne treuue pas en ma char que ie le puisse parfeire quar ie ne fais pas touz iours le bien don iay la uolente. mes fais aucune foiz le mal que ie refuse et que ie ne uueil. Et a la fin il con clut. ie uoi dit il une autre loy en mes membres con- traires et contredisant a la loy de ma pensee et de mon esperit. Et qui me meit en la seruitute chetiue de la loy de pechie. qui est en mõ corps et en mes membres. Las dolenz chetif hõme que ie suis et chetiue creature. qui sera celi qui me deliua- ra du corps de ceste mort. ¶Quar a la uerite nostre corps est la chartre de lame selonc ce que dit david en son psautier. Educ de car- cere animam meam. et c̾. ¶ Et enten diligẽment \textit{que} aucunes foiz il auient que de .ii. emprisonnez et enchar- trez. Lun a lissue de la char- tre est a mort mis et con- dempne. Et lautre a honeur est deliure. Si cõme il nous fu montre ou liure de gene- si. du meistre penetier du Roy pharaon et son boteil- lier. Des quiex le premier \marginpar{.xiii. }fu pendu et lautre resta- bli en son office. Selonc ce mesme que dit le sage ou liure de ecclesiastes. De carcere cathenis qz int̾ dum quis egreditur ad regnum. \textit{et} c̃. Tout ainsi est il des ames qui issent par mi la mort de la char- tre du corps quar aucu- nes sont a mort pardura- ble condempnees. Les au- tres en gloire et honeur \textit{per}durable esleuees. ¶En raconte. que il fu une no- ble dame iuene belle riche piteuse. deuote et de bonnes euures pleĩne. La quele de la uolente nostre seigneur fu ferue de meselerie tres horrible. La quele mala- die si paciẽment portoit et si ioieusement que il sembloit quelle se glori- .c.iiii. fiast auec. Saint pol en ses maladies. Auint que un esuesques oie la renõmee de sa bonte la uint uisiter de tel maladie si entachee et en leidie. en soy merueil- lant du diuin iugement par grant compassion prist a plorer. Et la dicte dame a rire et soy esleecier. Et quant elle uit lesuesque plorant se li demanda la cause de son pleur Respõ- di leuesque. que cestoit par pitie et par compassion. et leurs li demanda pour quoy elle auoit ris. Et elle li respondi en tel ma- niere. Sire dit la dame se uous esties enclos en une chartre de la quele uous ne porries iamais issir ius\textit{que}s a tant que les murs che- issent du tout en tout vouᷤ vit exemple nauriez iames ioie tant cõme uous uerries les murs de la dicte chartre fermes fors et entiers. Mes quant uous les uerriez petit \textit{et} petit de choair a bon droit uous feries ioye. quar uoꝰ auries ferme esperence de ur̃e brief deliurance. ¶Ain- si lame de moy desirrans issir de la chartre de ce po- ure corps quant elle con- sidere les paroiz du corps dedanz brief temps estre destruites. seioist pour ce quele a esperance tost estre de p̾sõ deliuree et deuant la face de son seigneur en gloire pardurable estre presentee lassaut du monde Apres lassaut et im- pugnacion de lene- mi de hõme et de la char di- sons et briefment de lassaut et impugnation du mon- de et des elemẽs ¶quar pour nous punir. La terre porte ronces espines et chardons couloures serpens boz et au- tres bestes ueninieuses. leau nous assaut et im- pugne par floz tempestes et granz rauines. qͥ sont aussi cõme deluges. par les quiex meint sont peril- lie. Ler par uenz tres uio- lens et tounoirres espoen- tanz le feu esparz fondres et autres espoentables im- pressions. ¶Et ainsi toute creature se combat contre les pecheeurs quar pour ce que nous abusons de toute creature par le iuge- ment iuste du createur no̾ sõmes contreint a sentir la uengence de dieu. La \textit{que}- le nous auons desseruie \marginpar{.xiiii. }en mont de manieres par noz pechiez. Et que il soit ainsi que nous soiens du monde et ou monde assailliz. a\textit{per}tement le dit nostre sauueeur en leuuã- gile saint Iehan. In mũ- do pressuram habebitis in me autem pacem. Ou mon- de dit il uous aurez de tur- bations et des mechiez meᷤ en moy uous trouuerez peiz. ¶Et pour ce dit .S̾. bernart. Le monde est. ou il a assez et a grant plante de mali- ce. et petit de sapience. ou quel toutes choses sont ui̾queuses. et po estables tout y est couuert de tene- bres. et plein de laz. Ou \textit{que}l les ames mont souuent perillent. les corps sont en grant affliction. Ou quel par tout uanite \textit{et} afflictiõ. .c.xvi. desperit. ¶ Item Saint ber- nart. Le peril du monde preu- ue. ce que po de gent le pue- ent eschaper. et pluseurs y perillent en la mer de mar- ceille a peĩnes. de .iiii. nes une perist. En la mer de ce mõ- de. a peĩnes de .iiii. ames une puet estre sauuee. ¶Et pour toutes ces choses nous devons estre humble et nous po prisier pechie orguiel. Chapitre .v. Apres la misere na- curele et contempo- rele couiẽt dire aucune cho- se de misere criminele de pe- chie. La quele senz nule cõ- paroison uaut pis \textit{que} toute̾ les autres senz la quele ne eussent onques este les autres. et de la quele toutes les autres naissent et uien- nent quar selonc le dit. S. gregoire. Se en nous ne regne pechie et iniquite. nuire ne nous puet adu̾- site. ¶ Et doiz sauoir que les mechiez et miseres des- seur dites apartiẽnent pͥn cipaument au corps et ren- dent le corps poure chetif \textit{et} miserable. Mes la misere de pechie a\textit{per}tenant a lame la rent et feit soubrete a toute maleuite. ¶ Et tout ainsi que la dignete de la- me. seurmonte de touz les corps qui furent. sont et seront la noblece. Aussi la misere de lame senz com- paroisson est plus uil et plus leide que la misere du corps queconques ele soi. ¶Otres grief necessi- te. o condicion de tres gͣnt maleurte qͣr auãt \textit{que} nous puissiens pechier nous sõ- mes ia en pechie enlacie. et o- bligie. auant que nous aiens feit bien ou mal \textit{nous} sõmes fil dire courpable de mort et enfant de perdi- tion. ¶ Et ceste misere \textit{nous} encorons nõ mie par nr̃e uolente ne par nostre pro- pre feit. Mes par le feit de nos premiers peres. ¶ Et ceste misere est nõmee des sains et des docteurs pechie originel. quar il nous ui- ent de nostre naissence. ¶De la quele misere puis- que par le sacrement de bap- tesme nous auons este de- liure. par le consentemẽt de nostre uolente. et par no̾ propres euures nous \textit{nous} metons a souueraĩne mi- sere. Et cest quant nous pechons mortelment. De ceste misere dit salmons en \marginpar{.xv. }prouerbes. Miseros facit populos peccatum. Pechie feit les hõmes poures che- tis et maleureus. ¶ Certes lame par pechie est tres horde leide uil et abhomi- nable. quar nule leidure nule corruption nule or- dure nest tant puent de- uant dieu et deuant les anges. Et nest nule cha- roigne tant soit horrible qui si pue deuant hõme. cõme pechie deuant dieu. Et pour ce pechie feit mõt a hair et refuser. ¶Et nõ mie seulement pͬ la cause dicte. Mes aussi pour ce que il rent lame feible \textit{et} enferme a tout bien. po- ure diseteuse et de tout biẽ despoillee et desnuee ¶Item par pechie est la parole de confession de honeste et de deuotion \textit{per}due et obliee. Et lame par pechie auu- glee. De dieu et des sains anges reprouee et a tour- menz \textit{per}durables deputee et obligee et de la gloire de paradis a touz iours hãnie et hors boutee. ¶De ceste misere parle dauid en son psautier. entel maniere. Iniquitates mee super gresse sunt caput meum \textit{et} c̾. Sequitur miser factus sum \textit{et} c̃. ¶ Et doiz noter. que tout aussi que celi est bien eureus qui a tout ce que il uuent et nul mal il ne uueut. Selonc la doc- trine saint augustin en lespitre que il feit a une da- me qui apelee estoit proba tout aussi est il mal eureus. et mal auentures qui ne puet auoir son uouloir \textit{et} .c.xiiii. uueut mal mont de cho- ses. ¶Tiex est le pecheeur. quar il uueut mont de choses a son tres gͣnt mal. Et ne desierre pas son biẽ ne bien honorable et hone- ste. ne profitable. mes son grant domage ne biẽ delita- ble. selon uerite et existen- ce combien que il semble estre delitable par dehors quant apparence. Et sou- uent uueut meĩtes choses que acomplir ne puet. ¶He diex cõme cest grant misere que de pecheeur. qui par soy puet cheoir et tre- buchier. et quant est de li il ne se puet releuer. Il se puet ordoier et nõ lauer naurer et nõ guerir occir- re et tuer et non uiuifier. Son mal uouloir et nõ mie son bien. Et se il auient aucune foiz que il uueille. ce que bon li est. nõ pourquant quant est de li il ne puet auoir son bon desir ne acomplir. ¶ De grant uolente le pe- cheeur court a sa mort. Et combien que clerement il uoie. son dampnement toute uoie ne le puet em- peschier. ¶ Donc dit saint bernart puis dit il que ie cõmencie a pechier. onques ne poy un iour passer \textit{que} ie ne pechasse. Et encor ne cesse ie de pechier. Et ain- si de iour en iour ie a iou- ste pechie a pechie. Et les uoi a mes yex senz en a- uoir doleur ie uoi choses tres honteuses. senz auoir doleur ie uoi choses tres honteuses senz auoir hon- te ne uergoigne en moy \marginpar{.xvi. }Ie uoi ce qui me deust estre cause et matiere de doleur. et au cuer nen sens nule doleur. Et le membre ma- lade qui sa doleur ne sent est mort. ou bien pres de la mort. Et la maladie que le malade ne sent est incura- ble. Ie suis dit il liez et dis soluz et point des pechiez que iay feiz ie ne mamã de ne corrige. A mes defauz don ie me suis confessez. touz les iours ie retourne. Et ne me garde pas de che- oir en la fosse en la quele ie suis cheuz autre foiz. et autrui hay ueu cheoir. Et ie qui plorer deusse pour les maus que iay feiz. \textit{et} pour pluseurs biens que ray leissie a feire. Las moy tout mest tourne a contraire \textit{que}r ie suis touz teues et refroi- diez de ferueur doroison. Et ia suis demorez touz froiz et senz sentement. Et pour ce ie ne puis plorer ne moy ne mes pechiez quar la fõ- taĩne de lermes que ie deus- se auoir est de\textit{per}tie de moy. ¶En quoi trouuõs \textit{nous} en pe- chie et en pecheeur autre mi- sere. quar aucunes foiz cui- des tu estre bons et tu es mau- uais. Sains et tu es mala- des. Et aucune foiz cuides estre sauuez et tu es a mort pardurable et a \textit{per}dicion par le souuerain iuge deputez et ordonnez. en lapocalipse est il escript. a leuesque de laodicie. Dicis quia di- ues sum et locupletatus nimis \textit{et} c̾. Tu diz ie suis mont riches et mont puis- senz et rien ne me faut. quar ie nay mestier de cõ- se il ne de doctrine. Et tu ne sez dit diex a cel euesque et a ceus qui sont desont de sa condicion tu ne sez \textit{que} tu es chetis miserables. tu es chetis et poures pour defaut de grace.Miserables pour la uilte de ta confusion po- ures. et senz merites de bon- nes euures. nuz quar tu es despoilliez de la uesteure de toute uertuz. Et auugles pour lignorance que tu as de ton estat. ¶ Et certeĩ- nement pechie qui feit la- me si miserable. a dieu hay- neuse et tres abhominable doit un chacuns souuerein- nement hair et fuir Esperi- alment princes granz seig- neurs et dames Ainsi le nous amoneste le sage roy salomon en ecclesiastique qui dit. Quasi a facie co- lubri fuge peccatum \textit{et} c̾. quar tout ainsi que nous natu- relment auons horreur de la colouure et la fuions pour la doubtance de son uenin. et fuions aussi les lyons pour peeur destre deuore. Et lespee de .ii. parz trenchant et bien amolue et afilee nous la doubtons. tout assi \textit{et} plus senz comparoison deuonons nous pechie doubter et hair. ¶Quar se il est ainsi que les paiens et les mescreenz ont pechie hay et fui pour les cau- ses dessus dictes especiaumẽt pour la uilte de lui et la lei- dure. par plus fort les creti- ens du precieus sanc nostre seigneur ihesucrist amoreuse- ment rachate. Et aus quiez diex a feit tant de benefices que nũs ne les porroit nõ- brer. ¶ Et que il soit ueritez \marginpar{.xvii. }.c.xxi. que les paiens. blasment pechie. en ten que dit tulles qui fu paiens. ou tiers li- ure que il feit des offices quar dit il nule chose de leidure nest a feire a hõme de bien suppose. que nũs ne le ueist. ¶Donc selonc ce que il dit philophie ensei- gne et amoneste que nule rien ne soit feite contre iu- stice contre continence et contre chaate. ¶ Et raconte ce mestre. que une fable fu trouuee de platon. dun hõ- me qui fu nõmez byses. le quel selonc les fables descendi en une grant fos- se en terre. Illec il trouua .i. cheual darem seur le quel cheual uit un hõme mõt grant et haut. Et uit un anel dor en son doy le quel il li osta et emporta. le quel anel estoit de tel uertu que quant la pierre de lanel estoit tournee par deuers la paume. il ueoit chascun et nũs ne le ueoit. Et qͣnt il remetoit lanel en son propre lieu de touz estoit ueuz. Cest hõme estoit pa- steur de bestes du roy du pays. ¶Et auint selonc ce que la fable raconte que ce byses par mi lanel prist la Royne a fẽme \textit{et} le Roy occist. Et ainsi par le benefice du dit anel en bri- ef temps il fu Roy des lyddiens. ¶ Or dit ce me- stre tullius. que se il estoit ainsi. que tu peusses auoir cel anel. pour ce ne deuroies tu pas pechier aussi po cõ- me se tu ne lauoies mie. quar selonc ce quil ꝯclut. hõme de bien quiert ce qͥ exemple est de honeste nõ mie chose celee et reprote. ¶Et sene- que qui fu paien. et mestre lempereeur neron. disoit en tel maniere. Nous nauõs onques assez tencie ne guerroie contre les uices. Aus quiex tu doiz \textit{per}secu- tion feire senz fin et senz mesure. ne fui. Et quecon- que chose soit qui ton cuer destruit dessire et dissipe. Se tu ne le po aies autrement de ton cuer traire ne oster tu deuroies auant ton cuer arachier. ¶ Or deuons donc pechie hayr .vi. Chapitre. Or y a encor une misere qui est ape- lee la misere denfer. qui mont feit a redoubter. De la quele dit dauid ou psau- tier. Cadent su\textit{per} eos car- bones in ignem deicies eos in miseriis non sub- sistent. Cest a dire que ca charbons ardenz charront seur les dampnez. Diex les metra et gitera ou feu den- fer. et auront tant de mi- sere que il ne la porront soustenir. ¶Par les char- bons poons nous enten- dre les corroz et les incre- pacions du souuerein iuge contre les dampnez par les quiex il seront confundu et leur consciences arses et brullees. par le feu ou quel il seront gitie est entendu le torment horrible du feu denfer. qui \textit{per}durablement les ames et les corps des tampnez tormentera. du quel dit ihesucrist en leuuã gile alez maudit de dieu mon pere ou feu denfer par- durable. qui est appareilliez \marginpar{.xviii. }Quar en pechie tu ne trou- ueras ne mesure ne fin. Math̾. xxv. au deable. et a ses anges. Et dit proprement dauid le prophete. que il seront mis en miseres Il ne dit pas en- misere mes en miseres qͣr il auront miseres que nũs ne porra nombrer penser ne raconter. ¶Mes cest grãt merueille que le prophete dit que en ces miseres il ne se pueent soustenir quar selonc la uerite les dampnez pardurablement uiuront et le tourment denfer soffre- ront et soutendront. ¶A ce en respont que a la uerite la ioustice de dieu requiert \textit{que} touz iours a souffrir tor- ment soient garde et senz fin soient tormente. et nõ pour quant pour la griefte de peĩnes il seront si affeibli que il semble que touz iourᷤ doient deffaillir leur uie dõc. nest autre chose que leur mort et toute uoie morir ne pueent. a ceste fin que leur iniquite soit senz fin punie. ¶Or considere donc tu qui es en grant seignorie. cõment cest chose horrible \textit{et} mont a redoubter cheoir en tel misere. la quele en ne puet souffrir et nõ pourqͣnt il la couuient touz iours durer senz faillir. ¶De la quele prioit estre deliurez iob. qui disoit. Dimitte me ut plangam paululũ dolorem meum. Et sensuit Terram miserie et tenebra- rum. Il uueut dire douz diex dõne moy espace de mes pe- chiez gemir et plorer auãt que il me couueigne tel me- chief endurer en la terre de misere et de tenebres. Le .vii. chapitre Puis que dit auons des miseres qͥ sont cõmunes a hõme et a fẽme briefment disons daucunes qui par especial touchent les fẽmes. a ceste fin queiles a preignent a garder et tenir humilite. Des quiex il est escript en genesi. Dixit do- minus ad mulierem mul- tiplicabo erumpnas tuas \textit{et} c̃. Et doiz sauoir que toutes les peĩnes don punie fu nostre premiere mere eue. apartiennent aus autres fẽmes. ¶Et apartient la p̾ miere peĩne au corps de la quele dit diex. Multiplica- bo erumpnas tuas. Ie mul- tipliray dit il tes enferme- tez. quar les fẽmes pour ce que naturelment elle sont plus feibles plus froides plus moites pour ce ont elles la complexion plus passible que nont les hom- mes. et pour ce sont soub- ietes a pluseurs enferme- tez. que ne sont les hõmes. ¶La seconde peĩne est quãt a son fruit et sa lignie en .ii. manieres. premierement quant au conceuement de lanfant. Et quant a ce diex dit Ie multiplieray tes con- ceuemenz. est a dire les pein- nes et miseres que tu as en conceuoir et enfanter. Et pius que la fẽme a- conceu. nous ueons sa fa- ce palir son uentre engros- sir. ses pas alentir son corps apesentir. sa uertu affeblir son repos amendrir. son cuer et sa pensee muer. son apetit chengier. et du peril de sa lignie et de sa \textit{per}sõne. mont forment douliter. \marginpar{.xix. }Secondement les fẽmes ont tres grant et angois- seuse peĩne a lanfantemẽt. Et quant a ce diex li dit. tu enfanteras tes enfans en doleur. Ceste doleur sen- tent elles plus par experiẽ- ce.que ne porroit expͥmer nostre eloquence. Donc cest une des gregneurs an- goisses que en puisse sen- tir en ce monde. Teles an- goisses sont consines et seurs a la mort. Son en- fant conceu ele porte nõ mie senz ennui et senz grant peeur. Ele enfante mes cest a grant tuistece et ci grant doleur. Lenfant ne ele norrist a grant soĩg et a grant labeur. Et si le garde a grant instance et aucune foiz a grant pleur ¶ La tierce peĩne des fẽmes est seruise et subiection. Et de ceste dit diex a eue. Sub uiri potestate eris. Ti seras dit il souz la poissence de lõme. et il aura seur toy seignorie. ¶ Sauoir deuõs que deuant le pechie de nos premiers peres lõme ser- uist sa fame et la fẽme sõ mari et son baron par ser- uise damistie et de charite. et de franche uolente. Mes apres pechie il fu impose ala fẽme en peĩne quele feust subiecte a son mari. aussi cõme par condicion dune seruitute. Et ce seuent par experience sensible au- cuneᷤ qui ont mariz mõt diuers et deguisiez qui leur meĩnent mont male uie. ¶ Et ainsi par toutes les choses dessus dictes il a\textit{per}t clerement la grandeur de la uilte et de la misere de condicion humeĩne quar hõme et fẽme de uil matie- re formez de plus uil est conceuz Et de tres horde ou corps de sa mere norriz. ¶Et aus si apert nostre misere mõt grant pour cause de nostre natiuite. quar guief est la misere naturele. plus gri- ef la contemporele. Tres gri- ef la misere de pechie que nous disons cumenele. ¶Et seur toutes les autres souuerẽinement grief et a redoubter la misere denfer que seuffrent les dampnez. la quele est perpetuele. Apres toutes ces miseres a touz cõmuneᷤ apert aucunes estre propres a fẽmes Toutes les queles choses diligẽment pesees et considerees est tout hõme contreint raisonna- blement soy humilier et de soy po sentir en cognoissent et recognoissent sa petitece. et que il est aussi cõme ne- ant. ¶ Et pour ce toute sa- ge dame queconqz eile soit doit considerer quel chose ele est quant a la condition de sa nature pour soy humi- lier. ¶ Et pour ce dit. Mi cheas le prophete. Humi liatio tua in medio tui. Aussi cõme se il uoussist dire en toy et de toy tu nas donc tu te doies esleuer ne orguillir. Ainz as matie- re de toy humilier quar humilite est ou milieu de toy. ¶ Et ce est la premiere conclusion de la premiere partie de ce liure qui nous aprent a humilier. ¶Da- uid dit ou psautier. Po- pulũ humilẽ saluũ fãcies. \marginpar{.xx. }.c.vi. Sire dit il le pueple hum- ble tu sauueras. Et ce \textit{nous} otroit ihesucrist qui uit et regne pardurablement. am̃. La seconde cõclusiõ de la p̃- miere \textit{par}tie est \textit{que} royne se doit auisier pour cause de ꝓmocion de sa fortune Ueu et considere com- ment sage dame doit regarder quele elle est quant a la condicion de sa nature. meintenant est a considerer quele est quant a la promocion de sa fortu- ne. la quele est dignite roy- al. De la quele bien garder elle se doit sagement pour uoaier et auisier quelle soit digne de tel dignete. ¶Quar royne doit auãt estre digne de royal digne- te que elle soit mise en tel dignete. Ou a tout le meins puis que eile est mise et esleuee en tel dignete elle doit soigneusement pour- chacier que digne soit da- uoir royal dignete. a ceste fin quele face a son estat honeur. et nõ pas lestat a li. quar autrement ce seroit metre la charue deuant les bues. Et ne seroit digne destre loee prisiee et honoree. mes deuroit estre deshonoree uituperee et reprouee quar de royal dignete ele aquer- roit son dampnement. ¶Et doit mont considerer \textit{et} peser cele qui royne est nõmee que ele na mie seulement ne ne doit auoir le non de royne. Mes doit auoir la realite la uerite et existence. cest a dire quele soit ueraiemẽt mement royne et nõ pas apelee royne tant seulem̃t pͥmier chap̾ 
\end{pages}
\end{document}
        